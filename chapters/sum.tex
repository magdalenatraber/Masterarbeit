\chapter{Summary}

14 pseudohalide complexes with 4-methoxypyridine and 4-hydroxy-methyl-pyridine as coligands were synthesized. This was conducted using the transition metals copper, zinc, cadmium and cobalt as metal center and the pseudohalides azide, rhodanide, dicyanamide and cyanate as ligands. Most  complexes possess an octahedral structure, two a tetrahedral and two had the rare coordination number five with a structure between square pyramidal and trigonal bipyramidal.\\ 

The crystal structure was solved via single crystal X-ray diffraction; the 4-methoxypyridine complexes are triclinic crystal systems (with the exception of an orthorombic one). One triclinic system aside, the 4-hydroxymethyl-pyridine complexes are monoclinic.\\

All 14 complexes were obtained from aqueous solutions, but water molecules  were not incorporated in the crystal structures, neither as aqua ligands nor as lattice water molecules. This is due to the two pyridine derivatives which are stronger ligands than water.\\

The five dicyanamide complexes form a structured series of polymeric chains (1D). In the tetrahedral zinc(II) and octahedral cobalt(II) mononuclear complexes the three-atomic pseudohalide anions \ce{N3^-} act as N-terminal ligands only. The copper azide complexes are polyhedral with the azide in bis-$\mu$-1,1-bridging mode  in the 4-MOP complex and bis-$\mu$-1,3-bridging mode in the 4-HOMP complex. \\

Among the four rhodanide complexes one possesses two hydroxymethylpyridines as bridging ligands, resulting in a dimeric structure and one with the \ce{SCN^-} ligand in bridging mode creating a 2D-sublattice. The other two are monomeric structures in which  \ce{SCN^-} is a terminal ligand (R-N-C-S). In one complex cyanate act as a pseudohalide ligand in an octahedral monomeric structure.\\

 Using UV-Vis spectroscopy, 3 absorption bands are observed for cobalt and one for copper-azide complexes. The colorless zinc and cadmium compounds  have no absorption bands in the visible and ultraviolett spectrum.\\ 


