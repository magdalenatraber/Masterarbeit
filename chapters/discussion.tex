\chapter{Discussion}
\section{Dicyanamide complexes}

Table \ref{tab:dcacomp} contains a list of in the CCDC published dicyanamide complexes \cite{ccdc} which possess a  similar ligand arrangement as the five newly synthesized dca complexes in table \ref{tab:mydca}. The central metal atom is coordinated to the terminal  N-atoms of 4 \ce{N#C-N^{-}-C#N} and to two other ligands in trans-configuration. In the new compounds these ligands are 4-methoxypyridine and 4-hydroxy-methyl-pyridine, whose N-atoms act as electron pair donor. The N-atoms on the ends of the dca are connected to the central metal atoms and make an end-to end-bridge between these. The metal doesn't bind to the middle N-atom of the dca ligand. The atom-atom-distances are  ranging from 2.0 to 2.5 \AA (M-N) and 1.1 to 1.3 (N-C). The angles are ranging from 174 to 177 $^\circ$ (N-C-N angle), 134 to 175 $^\circ$ (M-N-C), 90 to 180 $^\circ$ (N-M-N) and 114 to 121 (C-N-C).

\begin{table}[htpb!]
\centering
\captionabove{List of newly synthesized dca complexes}
\begin{tabular}{|l|}
\hline
Compound name\\
\hline
\ce{[Cd(dca)_2(4-methoxypyridine)_2]_n}\\
\hline
\ce{[Cu(dca)_2(4-methoxypyridine)_2]_n}\\
\hline
\ce{[Zn(dca)_2(4-methoxypyridine)_2]_n}\\
\hline
\ce{[Co(dca)_2(4-hydroxymethylpyridine)_2]_n}\\
\hline
\ce{[Cu(dca)_2(4-hydroxymethylpyridine)_2]_n}\\
\hline
\end{tabular}
\label{tab:mydca}
\end{table}



\begin{table}[htpb!]
\centering
\captionabove{List of structures of in the CCDC  published dca complexes \cite{ccdc}}
\begin{tabular}{|p{10cm}|l|}
\hline
Compound name &  Identifier\\
\hline
catena-(bis(($\mu_2$-Cyanocyanamidato-N1,N5)-(1H-imidazol-3-yl))-nickel) \cite{buhzua} & BUHZUA\\
\hline
catena-(bis($\mu_2$-Dicyanamide)-dipyridyl-manganese(II)) \cite{cerduy} & CERDUY\\
\hline
catena-(bis($\mu_2$-Dicyanamido)-bis(imidazole-N3)-cadmium) \cite{emimuj} & EMIMUJ\\
\hline
catena-(bis($\mu_2$-1,5-Dicyanamido)-bis(methanol)-copper(II)) \cite{hetree} & HETREE\\
\hline
catena-[tetrakis($\mu_2$-Dicyanamido)-bis(m2-4,4'-ditriazolylmethane)-di-manganese(II)] \cite{ipefak} & IPEFAK\\
\hline
catena-(bis($\mu_2$-Dicyanamido-N,N'')-bis(2-aminopyrimidine)-copper(II)) \cite{keqwim} & KEQWIM\\
\hline
catena-(bis(($\mu_2$-Cyanocyanamidato-N1,N5)-(1H-imidazol-3-yl))-cobalt) \cite{kuhxiv01} & KUHXIV01\\
\hline
catena-(bis($\mu_2$-Dicyanamide)-bis(3-cyanopyridine)-copper(II)) \cite{lapgal} & LAPGAL\\
\hline
catena-[bis($\mu_2$-dicyanamido)-bis(quinoxaline)-zinc] \cite{nodjat} & NODJAT\\
\hline
\end{tabular}
\label{tab:dcacomp}
\end{table}



\section{4-MOP  and 4-HOMP complexes}



This chapter refers to the publications in the CCDC regarding 4-MOP and 4- HOMP complexes shown in table \ref{tab:pubpy} (chapter \ref{ch:4HOMP}) and table \ref{tab:pubmepy} (chapter \ref{ch:4MOP}). \cite{ccdc} For 4-hydroxy-methyl-pyridine the most common central metal atoms were: copper (9 complexes), Pt (7 complexes), Ni (6 complexes) and Ag (5 complexes). The  4-methoxypyridine complexes offer a wider range in metal atoms like platinum, cobalt, ruthenium or rhenium, to name a few.  Other ligands used in these complexes were small ligands like chloride, rhodanide or nitrate and  big ligands such as other pyridines. The coordination number ranges from 2 to 6.\\
 Table \ref{tab:mypy} shows  the complexes synthesized for this work, of which nine were 4-MOP  and five were 4-HOMP complexes. The pseudohalides azide, cyanate, dicyanamide and rhodanide were used as ligands. Some of the ligands  like azide, dicyanamide, rhodanide and 4-hydroxymethyl-pyridine acted as  bridging ligands connecting metal centers. In the dimeric complex \ce{[Cu(NCS)2(4-hydroxymethylpyridine)2]2} the organic ligand act as N,O-bridge, in the other 13 structures of the master thesis the two pyridine derivative ligands function as N-terminal ligands only.



\begin{table}[htpb!]
\centering
\captionabove{Structures of newly synthesized 4-methoxypyridine and 4-hydoxymethylpyridine complexes}
\begin{tabular}{|l|}
\hline
\ce{[Co(N_3)_2(4-methoxypyridine)_4]}\\
\hline
\ce{[Cu(N_3)_2(4-methoxypyridine)_2]_n}\\
\hline
\ce{[Zn(N_3)_2(4-methoxypyridine)_2]}\\
\hline
\ce{[Co(OCN)_2(4-methoxypyridine)_4]}\\
\hline
\ce{[Cd(dca)_2(4-methoxypyridine)_2]_n}\\
\hline
\ce{[Cu(dca)_2(4-methoxypyridine)_2]_n}\\
\hline
\ce{[Zn(dca)_2(4-methoxypyridine)_2]_n}\\
\hline
\ce{[Co(SCN)_2(4-methoxypyridine)_4]}\\
\hline
\ce{[Cu(SCN)_2(4-methoxypyridine)_2]_n}\\
\hline
\ce{[Cu(N_3)_2(4-hydroxymethylpyridine)]_n}\\
\hline
\ce{[Co(dca)_2(4-hydroxymethylpyridine)_2]_n}\\
\hline
\ce{[Cu(dca)_2(4-hydroxymethylpyridine)_2]_n}\\
\hline
\ce{[Cu(SCN)_2(4-hydroxymethylpyridine)_2]_2}\\
\hline
\ce{[Zn(SCN)_2(4-hydroxymethylpyridine)_2]}\\
\hline
\end{tabular}
\label{tab:mypy}
\end{table}