\part{Experimental}
\chapter{Preparation and technical data}
\section{General overview}
The chemicals used in the synthesis of the complexes were provided by Merck. No further purification was necessary. Destilled water was used as solvent for all reactions. All chemicals were handled with care and only small doses were used. For the reactions 100 mL flasks with screw caps were used. The solutions were  heated in waterbaths standing on heating plates. For a  slow cooling to room temperature the heating plates were shut off and the flasks were left in the waterbaths until 20$^\circ$C were reached. If no crystals were obtained the screw caps were removed, so the water could slowly evaporate. The crystalls were obtained using filtration under reduced pressure.\\
The IR spectra of the dca complexes were compared with the IR-spectrum of sodium dicyanamide in the attachment p. \pageref{fig:dca-ir}, the other complexes (azide, cyanate and rhodanide) were compared with literature. \cite{kazuo} 
\section{Used devices and programs}
For single crystal X-ray measurement the following devices were used:
A Bruker and APEX II CCD diffractometer (MoK$\alpha$ radiation $\lambda$ = 0.71073 \AA) with $\omega$-scan mode and
graphite-monochromator at 100K. Data was collected and processed using APEX and SAINT \cite{apex} software packages. All data was corrected, for absorption  Laue symmetry requirements were applied. \cite{laue}
The SHELXTL/PC program package \cite{shelxtl} was applied for  structure solution (direct methods) and structure refinement (least squares). PLATON \cite{platon} (a program for automated
calculation of derived geometrical data) was used for supporting the structure solution process.
Additionally a Bruker-AXS SMART APEX CCD diffractometer at 100K with Mo-radiation ($\lambda$ = 0.7107 \AA) and
graphite monochromator was used.
Data was collected by Assoc.Prof. Dipl.-Ing. Dr.techn. Roland Fischer and structure refinement was done by Ao. Univ.-Prof. Dr. Franz A. Mautner.
\\For IR-analysis an Alpha-P spectrometer by Bruker was provided. It was used to characterize specific bands from pyridines and pseudohalides. The spectra were measured in the range of  400 cm$^{-1}$ to 4000 cm$^{-1}$ with the program Opus. \cite{opus}
\\The UV-VIS-measurements were conducted using a Lambda 950 UV/ VIS/ NIR spectrometer by Perkin Elmer. Measurement of the spectra ranged from 200 to 2500 nm. The program for processing and data collection was UVWINLab software. \cite{uvwin}